\documentclass{article}
\date{\today}
\title{Working title}
\author{Not-working authors}

\begin{document}
\maketitle
\abstract{ }
\section{Introduction}
Shapes/objects/images can be acted on by simple transformations that change
their appearance. We want to recognise the objects or identify the set of
transformations. If the set is a group, then we can seek to either identify
the group or solve the equivalence problem by
\begin{enumerate}
    \item Registration, or
    \item Invariance.
\end{enumerate}
The first is commonly studied but is computationally expensive and prone to
local minima. If you're not interested in the actual transformation it is
therefore to be avoided. Here we'll consider invariance. There are many
choices of type e.g. Fourier, integral, or differential invariants. The
last set are cheap to compute and have nice theoretical properties but are
not robust to noise. However, we'll use them anyway. This paper is a
tutorial on differential invariants for planar Lie groups (e.g.
Euclidean, affine, projective) since they are relatively simple, useful,
and relevant to images. We'll use these to demonstrate the methods of
constructing and using differential invariant signatures. Note that this
reduces the recognition problem to recognising higher-dimensional
differential invariant signatures. We'll mention some approaches to this.
See the companion paper for an implementation.

\subsection{Tutorial on differential invariants / differential geometry}
\subsubsection{Differential geometry stuff}
\begin{description}
    \item[Completeness]
    \item[Invariant]
    \item[joint invariant and weight]
    \item[transvectant]
    \item[generic functions]
    \item[differential invariant order]
    \item[transitive]
    \item[free]
    \item[effective]
    \item[group prolongation]
\end{description}

\subsubsection{Construction of signatures}
Including dealing with projection to deal with scalings
Signatures are higher dimensional objects 
Completeness vs practicality

Will include Translation as basic example.

\section{Classical Invariant Theory}
\subsection{Description of method}
\subsection{Groups}
\begin{itemize}
    \item $SE(2)$
    \item $E(2)$
    \item $Sim(2)$
    \item $SA(2)$
    \item $A(2)$
\end{itemize}

\section{The Moving Frame Method}
\subsection{Description of method}
\subsection{Groups}
\begin{itemize}
    \item M\"obius
    \item Projective
\end{itemize}

\subsection{Diffeomorphism groups}
Diff then Diff vol and Diff conf. No invariants -> there are some.

\section{Experiments}
Effects of noise
\subsection{Example signatures}
\subsection{`Spot the group'}

\section{Discussion and Conclusions}
\subsection{Colour images}
\subsection{Number of invariants}
\end{document}
