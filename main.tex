\documentclass[11pt]{article}
% mathtools for \coloneqq
\usepackage{amsmath, amssymb, mathtools}

\title{Invariant signatures}
\date{\today}
\author{Huey, Duey, and Louey}

\begin{document}
\maketitle

\section{Introduction}
\subsection{Notation}
Let $G$ be a planar transformation group that acts on $k$-colour images $f
\colon \mathbb{R}^2 \rightarrow \mathbb{R}^k$ by $\varphi \cdot f \coloneqq
f \circ \varphi^{-1}$, where $\varphi \in G$.

When referring to coordinates, with $x \in \mathbb{R}^2$, we write $\hat{x}
= \varphi(x)$  and define the image post-transformation in the new
coordinates as
\begin{equation}
  \hat{f}(\hat{x}) = f(x).
\end{equation}
We derive relationships between the derivatives before and after
transformation by the chain rule:
\begin{equation*}
  f_{, i} = \hat{f}_{,k} \varphi_{k, i} 
\end{equation*}
etc. 





\section{The Stamp Collection I: The Affine group and its subgroups}
In this section we consider all the groups whose transformation is of the
form. For these groups, the change of variables formulae are, in tensor,
and matrix notations,
\begin{align}
  f_{,i} &= \hat{f}_{, k} a_{ki} & \nabla f &= A^T \nabla \hat{f} \\
  f_{,ij} &= \hat{f}_{, kl} a_{ki} a_{lj} & 
  \nabla^2 f &= A^T \nabla^2 \hat{f} A
\end{align}
$\varphi(x) = Ax + b$ for some matrix $A$ and vector $b$.

\subsection{SE(2)}
In this case we have $A^TA = I$ and $\det A = 1$.


\end{document}
